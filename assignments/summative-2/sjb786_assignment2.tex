\documentclass{article}
\usepackage{amsmath, amssymb}
\title{Week 12 Continuous Assessment}
\author{1803086}
\begin{document}
\maketitle
\begin{enumerate}
  \item

  \item
        \[
        d(x,y,z) = 1 \Longleftrightarrow x = y = 1 \text{ or } x = z = 0
        \]

        \begin{enumerate}
          \item
                \begin{align*}
                  d(x,y,z) = &(x \wedge y) \vee (\neg x \wedge \neg z) \\
                  &\equiv \neg ( \neg ( x \wedge y ) \wedge \neg (\neg x \wedge \neg z) ) \\
                  &\equiv \neg ((\neg x \vee \neg y) \wedge (x \vee z)) \\
                  &\equiv \neg ((\neg x \wedge x) \vee (\neg x \wedge z) \vee (\neg y \wedge x) \vee (\neg y \wedge z)) \\
                  &\equiv \neg (\neg x \wedge z) \wedge \neg ( \neg y \wedge x ) \wedge \neg (\neg y \wedge z) \\
                  &\equiv (x \vee z) \wedge (y \vee x) \wedge (y \vee \neg z)
                \end{align*}

          \item
                \begin{enumerate}

                  \item
                        \begin{align*}
                          x \vee y &\Longleftrightarrow \\
                          & d(0, && d(\\
                                && d(\neg x,\neg y,d(x,x,x)),\\
                                && d(\neg x,\neg y,d(x,x,x)),\\
                                && d(\neg x,\neg y,d(x,x,x)) \\
                          &&) \\
                         &,d(\neg x,\neg y,d(x,x,x)))
                        \end{align*}
                        Where:
                        \begin{itemize}
                          \item $\neg x = d(0,d(x,x,x),x)$
                          \item $\neg y = d(0,d(y,y,y),y)$
                        \end{itemize}

           \item $x \wedge y \Longleftrightarrow d(x,y,d(x,x,x))$
         \end{enumerate}


        \end{enumerate}
  \item By the change-of-basis theorem

        Let the circuits $\Omega_{1}$ defined using the operators $\{ \neg , \wedge , \vee \} $ and let the circuits defined over $\{ d, 0 \} $ be the set $\Omega_{2}$

        for each circuit $c_{i} \in \Omega_{1}$ let there be an equivalent circuit $c'_{i} \in \Omega_{2}$ of size $s_{i}$ and depth $l_{i}$ which computes $c_{i}$. Also let $s = \max_{i} s_{i}$ and $l = \max_{i} l_{i}$.

        Given a circuit $C \in \Omega_{1}$ , we can construct a logically equivalent circuit $C' \in \Omega_{2}$ by replacing all non-source nodes labelled with $c_{i}$ by the circuit $c'_{i}$.

        With $\Omega_{1} = \{ \neg , \vee , \wedge\} $ and $\Omega_{2} = \{ 0, d \} $ we can show that every $\Omega_{1}$ circuit $C $ can be polynomially transformed to a $\Omega_{2}$ circuit $C'$.

        for the operator \(\neg\), we have seen in (b) that this can be transformed into $d(0,d(x,x,x),x)$. This node can therefore be replaced with a circuit of depth 6.

        The operator $\vee$ has been shown to be equivalent to the \(\Omega_{2}\) circuit of depth 9.

        The \(\wedge\) operator has been shown to be equivalent to a \(\Omega_{2}\) circuit of depth 6.

        Therefore we can see that in this case $l = 9$ and so the depth of $C'$ is at most $9 \cdot \texttt{depth} (C)$ which is in $O(\texttt{depth} (C))$


\end{enumerate}

\end{document}
